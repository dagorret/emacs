% Created 2025-11-24 lun 00:24
% Intended LaTeX compiler: pdflatex
\documentclass[11pt]{article}
\usepackage[utf8]{inputenc}
\usepackage[T1]{fontenc}
\usepackage{graphicx}
\usepackage{longtable}
\usepackage{wrapfig}
\usepackage{rotating}
\usepackage[normalem]{ulem}
\usepackage{amsmath}
\usepackage{amssymb}
\usepackage{capt-of}
\usepackage{hyperref}
\author{Carlos}
\date{\today}
\title{Documentación de atajos y comandos Emacs}
\hypersetup{
 pdfauthor={Carlos},
 pdftitle={Documentación de atajos y comandos Emacs},
 pdfkeywords={},
 pdfsubject={},
 pdfcreator={Emacs 29.3 (Org mode 9.6.15)}, 
 pdflang={Spanish}}
\begin{document}

\maketitle
\tableofcontents


\section{Introducción}
\label{sec:org00f8e16}
Este archivo documenta:

\begin{itemize}
\item Los atajos de teclado \textbf{básicos} de Emacs (los más usados para movimiento,
cortar/pegar, ventanas, archivos, etc.).
\item Todos los atajos \textbf{personalizados} definidos en mi configuración:
\begin{itemize}
\item \texttt{core-base.el}
\item \texttt{core-files.el}
\item \texttt{core-ui.el}
\item \texttt{core-dev.el}
\item \texttt{core-notes.el}
\item \texttt{lang-org.el}
\item \texttt{lang-writing.el}
\item \texttt{lang-prog.el}
\item \texttt{lang-web.el}
\item \texttt{lang-web-extras.el}
\end{itemize}
\item Las funciones \texttt{my/...} que agregan comportamiento nuevo, con su código
fuente en bloques \texttt{emacs-lisp}.
\end{itemize}

Puedo abrir este archivo dentro de Emacs y usarlo como manual rápido.

\section{Convenciones de teclas}
\label{sec:orgd6f2241}

\begin{description}
\item[{\texttt{C-}}] tecla Control
\item[{\texttt{M-}}] tecla Meta (Alt o \texttt{ESC})
\item[{\texttt{S-}}] tecla Shift
\item Ejemplos:
\begin{itemize}
\item \texttt{C-x C-f}  → Control + x, luego Control + f
\item \texttt{M-x}      → Meta (Alt) + x
\item \texttt{<f6>}     → tecla de función F6
\end{itemize}
\end{description}

\section{Movimiento, edición, cortar y pegar (por defecto)}
\label{sec:orge39f18d}

\subsection{Movimiento del cursor}
\label{sec:org56cb0d8}

\begin{center}
\begin{tabular}{ll}
Tecla & Acción\\[0pt]
\hline
C-f & Carácter adelante\\[0pt]
C-b & Carácter atrás\\[0pt]
C-n & Línea siguiente\\[0pt]
C-p & Línea anterior\\[0pt]
M-f & Palabra adelante\\[0pt]
M-b & Palabra atrás\\[0pt]
C-a & Inicio de línea\\[0pt]
C-e & Fin de línea\\[0pt]
M-a & Inicio de oración\\[0pt]
M-e & Fin de oración\\[0pt]
M-< & Principio del buffer\\[0pt]
M-> & Final del buffer\\[0pt]
C-v & Página abajo\\[0pt]
M-v & Página arriba\\[0pt]
\end{tabular}
\end{center}

\subsection{Selección, cortar, copiar, pegar}
\label{sec:org09a08d3}

\begin{center}
\begin{tabular}{ll}
Tecla & Acción\\[0pt]
\hline
C-SPC & Empezar a marcar región\\[0pt]
C-g & Cancelar región/comando\\[0pt]
C-w & Cortar (kill) región\\[0pt]
M-w & Copiar región\\[0pt]
C-y & Pegar (yank)\\[0pt]
M-y & Navegar por el kill-ring\\[0pt]
\end{tabular}
\end{center}

\subsection{Deshacer}
\label{sec:org777e709}

\begin{center}
\begin{tabular}{ll}
Tecla & Acción\\[0pt]
\hline
C-/ & Deshacer\\[0pt]
C-\_ & Deshacer\\[0pt]
\end{tabular}
\end{center}

\section{Archivos y buffers}
\label{sec:org95f46e6}

\subsection{Archivos}
\label{sec:org269ac6d}

\begin{center}
\begin{tabular}{ll}
Tecla & Acción\\[0pt]
\hline
C-x C-f & Abrir archivo\\[0pt]
C-x C-s & Guardar archivo\\[0pt]
C-x C-w & Guardar como…\\[0pt]
C-x C-v & Reemplazar buffer por otro\\[0pt]
\end{tabular}
\end{center}

\subsection{Buffers}
\label{sec:orge822816}

\begin{center}
\begin{tabular}{ll}
Tecla & Acción\\[0pt]
\hline
C-x b & Cambiar de buffer\\[0pt]
C-x C-b & Lista de buffers\\[0pt]
C-x k & Cerrar (kill) buffer actual\\[0pt]
\end{tabular}
\end{center}

\section{Ventanas (splits)}
\label{sec:orgf4a9f03}

\begin{center}
\begin{tabular}{ll}
Tecla & Acción\\[0pt]
\hline
C-x 1 & Dejar solo la ventana actual\\[0pt]
C-x 2 & Dividir horizontalmente (una arriba/otra)\\[0pt]
C-x 3 & Dividir verticalmente (izquierda/derecha)\\[0pt]
C-x 0 & Cerrar ventana actual\\[0pt]
C-x o & Ir a la otra ventana\\[0pt]
\end{tabular}
\end{center}

\section{Búsqueda y reemplazo}
\label{sec:orgf776ec5}

\begin{center}
\begin{tabular}{ll}
Tecla & Acción\\[0pt]
\hline
C-s & Búsqueda incremental adelante\\[0pt]
C-r & Búsqueda incremental atrás\\[0pt]
M-\% & Buscar y reemplazar (query-replace)\\[0pt]
C-M-\% & Buscar y reemplazar con regexp\\[0pt]
\end{tabular}
\end{center}

\section{Ayuda en Emacs}
\label{sec:org899d823}

\begin{center}
\begin{tabular}{ll}
Tecla & Acción\\[0pt]
\hline
C-h k & Describir qué hace una tecla\\[0pt]
C-h f & Describir función\\[0pt]
C-h v & Describir variable\\[0pt]
C-h b & Ver todos los atajos activos en el buffer\\[0pt]
C-h m & Ayuda de los modos activos\\[0pt]
C-h t & Tutorial interactivo de Emacs\\[0pt]
\end{tabular}
\end{center}

\section{Dired (gestor de archivos)}
\label{sec:org8f1fe28}

En \texttt{Dired} se agregan atajos personalizados, pero primero los básicos:

\begin{center}
\begin{tabular}{ll}
Tecla & Acción\\[0pt]
\hline
RET & Abrir archivo / entrar directorio\\[0pt]
\^{} & Subir al directorio padre\\[0pt]
+ & Crear directorio\\[0pt]
m & Marcar archivo\\[0pt]
u & Desmarcar archivo\\[0pt]
d & Marcar para borrar\\[0pt]
x & Ejecutar borrados marcados\\[0pt]
g & Refrescar listado\\[0pt]
\end{tabular}
\end{center}

Los atajos extra definidos en mi configuración están documentados más abajo.

\section{Org mode básico}
\label{sec:org0c649f3}

\begin{center}
\begin{tabular}{ll}
Tecla & Acción\\[0pt]
\hline
TAB & Expandir/colapsar bloque\\[0pt]
S-TAB & Ciclar visibilidad global\\[0pt]
M-RET & Nuevo ítem al mismo nivel\\[0pt]
M-<up> & Subir encabezado\\[0pt]
M-<down> & Bajar encabezado\\[0pt]
M-<left> & Promover encabezado\\[0pt]
M-<right> & Degradar encabezado\\[0pt]
\end{tabular}
\end{center}

TODOs y agenda:

\begin{center}
\begin{tabular}{ll}
Tecla & Acción\\[0pt]
\hline
C-c C-t & Cambiar TODO/DONE\\[0pt]
C-c . & Insertar fecha\\[0pt]
C-c C-s & SCHEDULED\\[0pt]
C-c C-d & DEADLINE\\[0pt]
\end{tabular}
\end{center}

\section{Atajos personalizados globales}
\label{sec:orgd78b4c7}

Los siguientes atajos se definen en mis módulos \texttt{core-*} y \texttt{lang-*}.

\subsection{Terminal integrada}
\label{sec:org3b954b6}

\begin{center}
\begin{tabular}{lll}
Tecla & Comando & Descripción\\[0pt]
\hline
C-c t & \texttt{vterm} & Abrir terminal \texttt{vterm}\\[0pt]
\end{tabular}
\end{center}

\subsection{Layout de desarrollo}
\label{sec:org44fe6c9}

\begin{center}
\begin{tabular}{lll}
Tecla & Comando & Descripción\\[0pt]
\hline
<f9> & \texttt{my/dev-layout} & Layout con Treemacs a la izquierda, código y vterm\\[0pt]
\end{tabular}
\end{center}

\subsection{Org-roam (notas diarias)}
\label{sec:orga649119}

\begin{center}
\begin{tabular}{lll}
Tecla & Comando & Descripción\\[0pt]
\hline
C-c n d & \texttt{org-roam-dailies-goto-today} & Ir a la nota diaria de hoy\\[0pt]
C-c n j & \texttt{org-roam-dailies-capture-today} & Capturar entrada en el diario\\[0pt]
\end{tabular}
\end{center}

\subsection{Org agenda y capture}
\label{sec:orgca386f6}

\begin{center}
\begin{tabular}{lll}
Tecla & Comando & Descripción\\[0pt]
\hline
C-c a & \texttt{org-agenda} & Agenda de Org\\[0pt]
C-c c & \texttt{org-capture} & Capturas rápidas de Org\\[0pt]
\end{tabular}
\end{center}

\subsection{Compilar y ejecutar C/C++}
\label{sec:orgde8dfc0}

\begin{center}
\begin{tabular}{lll}
Tecla & Comando & Descripción\\[0pt]
\hline
C-c r & \texttt{my/compile-and-run} & Compilar y ejecutar archivo actual\\[0pt]
\end{tabular}
\end{center}

\subsection{Diccionarios de ortografía}
\label{sec:orgd961ff8}

\begin{center}
\begin{tabular}{lll}
Tecla & Comando & Descripción\\[0pt]
\hline
C-c d s & \texttt{my/dict-spanish} & Diccionario español (es\textsubscript{AR})\\[0pt]
C-c d e & \texttt{my/dict-english} & Diccionario inglés (en\textsubscript{US})\\[0pt]
\end{tabular}
\end{center}

\subsection{Exportar a DOCX}
\label{sec:org407c1fa}

\begin{center}
\begin{tabular}{lll}
Tecla & Comando & Descripción\\[0pt]
\hline
C-c e w & \texttt{my/export-buffer-to-docx} & Exportar buffer actual a DOCX usando pandoc\\[0pt]
\end{tabular}
\end{center}

\section{Atajos personalizados en Dired}
\label{sec:org3798b1c}

En \texttt{core-files.el} se agregan atajos para crear directorios y archivos
directamente desde Dired:

\begin{center}
\begin{tabular}{llll}
Modo & Tecla & Comando & Descripción\\[0pt]
\hline
Dired & <f6> & \texttt{my/dired-create-empty-file} & Crear archivo vacío en Dired\\[0pt]
Dired & <f7> & \texttt{my/dired-create-directory} & Crear nuevo directorio en Dired\\[0pt]
\end{tabular}
\end{center}

\section{Atajos personalizados en \LaTeX{}}
\label{sec:orgaf2ff31}

En \texttt{lang-writing.el} (AUCTeX / \LaTeX{}):

\begin{center}
\begin{tabular}{llll}
Modo & Tecla & Comando & Descripción\\[0pt]
\hline
\LaTeX{}-mode & C-c e & \texttt{my/latex-equation} & Insertar entorno \texttt{equation}\\[0pt]
\LaTeX{}-mode & C-c f & \texttt{my/latex-figure} & Insertar entorno \texttt{figure}\\[0pt]
\LaTeX{}-mode & C-c t & \texttt{my/latex-table} & Insertar entorno \texttt{table}\\[0pt]
\LaTeX{}-mode & C-c i & \texttt{my/latex-itemize} & Insertar entorno \texttt{itemize}\\[0pt]
\end{tabular}
\end{center}

\section{Funciones \texttt{my/...} y código fuente}
\label{sec:org5c45065}

A continuación, el código fuente de todas las funciones \texttt{my/...} definidas en
los módulos. Sirve como referencia y también como ejemplo para futuras
modificaciones.

\subsection{\texttt{core-dev.el} :: layout de desarrollo}
\label{sec:orgb26d0a7}

\begin{verbatim}
(defun my/dev-layout ()
  "Abrir layout con Treemacs a la izquierda, código arriba y vterm abajo."
  (interactive)
  (delete-other-windows)
  ;; Panel izquierdo: Treemacs
  (treemacs)
  ;; Nos movemos a la ventana de la derecha para código
  (select-window (next-window))
  ;; Partimos la derecha en dos (arriba código, abajo terminal)
  (split-window-below)
  ;; Ventana superior derecha: se queda para el buffer actual
  (other-window 1)
  ;; Ventana inferior derecha: vterm
  (vterm)
  ;; Volver a la ventana de código
  (other-window -1))
\end{verbatim}

\subsection{\texttt{core-files.el} :: utilidades para Dired}
\label{sec:org2307cc6}

\begin{verbatim}
(defun my/dired-create-directory ()
  "Crear un nuevo directorio en el Dired actual, con un prompt claro."
  (interactive)
  (let* ((dir (dired-current-directory))
         (name (read-string (format "Nombre del nuevo directorio en %s: " dir))))
    (when (and name (not (string-empty-p name)))
      (let ((full (expand-file-name name dir)))
        (make-directory full t)
        (revert-buffer)
        (message "Directorio creado: %s" full)))))

(defun my/dired-create-empty-file ()
  "Crear un nuevo archivo vacío en el Dired actual, con un prompt claro."
  (interactive)
  (let* ((dir (dired-current-directory))
         (name (read-string (format "Nombre del nuevo archivo en %s: " dir))))
    (when (and name (not (string-empty-p name)))
      (let ((full (expand-file-name name dir)))
        (with-temp-buffer
          (write-file full))
        (revert-buffer)
        (message "Archivo creado: %s" full)))))

(with-eval-after-load 'dired
  (define-key dired-mode-map (kbd "<f6>") #'my/dired-create-empty-file)
  (define-key dired-mode-map (kbd "<f7>") #'my/dired-create-directory))
\end{verbatim}

\subsection{\texttt{lang-prog.el} :: programación (C/C++)}
\label{sec:orgf9907e3}

\begin{verbatim}
(defun my/c-enabled ()
  "Config básica para C/C++: estilo, sangría, etc."
  (c-set-style "linux")
  (setq c-basic-offset 4
        indent-tabs-mode nil))

(add-hook 'c-mode-hook #'my/c-enabled)
(add-hook 'c++-mode-hook #'my/c-enabled)

(defun my/compile-and-run ()
  "Compilar y ejecutar el archivo C/C++ actual en una ventana vterm.
Compila con g++ -std=c++20 -O2 y ejecuta el binario resultante."
  (interactive)
  (unless buffer-file-name
    (user-error "El buffer no está asociado a un archivo"))
  (save-buffer)
  (let* ((src buffer-file-name)
         (exe (file-name-sans-extension src))
         (cmd (format "g++ -std=c++20 -O2 -o %s %s && %s\n" exe src exe)))
    (unless (executable-find "g++")
      (user-error "g++ no está instalado"))
    (let ((vterm-buf (get-buffer "*vterm-compilacion*")))
      (if vterm-buf
          (pop-to-buffer vterm-buf)
        (setq vterm-buf (vterm "*vterm-compilacion*")))
      (vterm-send-string cmd)
      (vterm-send-return))))
\end{verbatim}

\subsection{\texttt{lang-web-extras.el} :: helpers para Vue / Alpine, etc.}
\label{sec:org72a6c6a}

\begin{verbatim}
(defun my/vue-insert-sfc ()
  "Insertar skeleton básico de Single File Component Vue 3."
  (interactive)
  (insert "<template>\n  <div class=\"\">\n  </div>\n</template>\n\n")
  (insert "<script setup>\n\n</script>\n\n")
  (insert "<style scoped>\n\n</style>\n")
  (message "Vue SFC insertado."))

(defun my/alpine-insert-component ()
  "Insertar un snippet básico para un componente con Alpine.js."
  (interactive)
  (insert "<div x-data=\"{ open: false }\">\n")
  (insert "  <button @click=\"open = !open\">Toggle</button>\n")
  (insert "  <div x-show=\"open\">\n")
  (insert "    Contenido...\n")
  (insert "  </div>\n")
  (insert "</div>\n")
  (message "Snippet Alpine.js insertado."))

(defun my/tailwind-insert-container ()
  "Insertar un contenedor básico con clases de Tailwind CSS."
  (interactive)
  (insert "<div class=\"max-w-4xl mx-auto px-4 sm:px-6 lg:px-8\">\n")
  (insert "  <!-- Contenido -->\n")
  (insert "</div>\n")
  (message "Contenedor Tailwind CSS insertado."))
\end{verbatim}

\subsection{\texttt{lang-writing.el} :: escritura, Markdown, \LaTeX{}, ortografía, exportación}
\label{sec:org91071a5}

Incluye helpers para:

\begin{itemize}
\item Cambiar diccionarios (español/inglés).
\item Helpers de Markdown (negrita, itálica, código, enlaces, etc.).
\item Helpers de \LaTeX{} (entornos \texttt{equation}, \texttt{figure}, \texttt{table}, \texttt{itemize}).
\item Exportar el archivo actual a DOCX con \texttt{pandoc}.
\end{itemize}

Código completo:

\begin{verbatim}
(defun my/dict-spanish ()
  "Cambiar diccionario a español (es_AR)."
  (interactive)
  (ispell-change-dictionary "es_AR")
  (message "Diccionario cambiado a español (es_AR)"))

(defun my/dict-english ()
  "Cambiar diccionario a inglés (en_US)."
  (interactive)
  (ispell-change-dictionary "en_US")
  (message "Dictionary changed to English (en_US)"))

(global-set-key (kbd "C-c d s") #'my/dict-spanish)
(global-set-key (kbd "C-c d e") #'my/dict-english)

;; Helpers Markdown
(defun my/markdown-wrap (left right)
  "Envolver la región activa (o palabra actual) entre LEFT y RIGHT."
  (if (use-region-p)
      (let ((beg (region-beginning))
            (end (region-end)))
        (goto-char end)
        (insert right)
        (goto-char beg)
        (insert left))
    (let ((bounds (bounds-of-thing-at-point 'word)))
      (when bounds
        (goto-char (cdr bounds))
        (insert right)
        (goto-char (car bounds))
        (insert left)))))

(defun my/markdown-bold ()
  "Negrita Markdown para región o palabra actual."
  (interactive)
  (my/markdown-wrap "**" "**"))

(defun my/markdown-italic ()
  "Itálica Markdown para región o palabra actual."
  (interactive)
  (my/markdown-wrap "*" "*"))

(defun my/markdown-inline-code ()
  "Código en línea Markdown para región o palabra actual."
  (interactive)
  (my/markdown-wrap "`" "`"))

(defun my/markdown-link ()
  "Insertar un enlace Markdown. Si hay región, usarla como texto."
  (interactive)
  (let* ((url (read-string "URL: "))
         (text (if (use-region-p)
                   (buffer-substring-no-properties (region-beginning)
                                                   (region-end))
                 (read-string "Texto del enlace: "))))
    (when (use-region-p)
      (delete-region (region-beginning) (region-end)))
    (insert (format "[%s](%s)" text url))))

(defun my/markdown-setup-keys ()
  "Configurar atajos para helpers de Markdown."
  (local-set-key (kbd "C-c m b") #'my/markdown-bold)
  (local-set-key (kbd "C-c m i") #'my/markdown-italic)
  (local-set-key (kbd "C-c m c") #'my/markdown-inline-code)
  (local-set-key (kbd "C-c m l") #'my/markdown-link))

(add-hook 'markdown-mode-hook #'my/markdown-setup-keys)

;; Helpers LaTeX
(defun my/latex-insert-environment (name)
  "Insertar un entorno LaTeX \\begin{NAME} ... \\end{NAME}, con punto de inserción en medio."
  (interactive "sNombre del entorno: ")
  (insert (format "\\begin{%s}\n" name))
  (save-excursion
    (insert (format "\n\\end{%s}\n" name))))

(defun my/latex-equation ()
  "Insertar entorno equation."
  (interactive)
  (my/latex-insert-environment "equation"))

(defun my/latex-itemize ()
  "Insertar entorno itemize."
  (interactive)
  (my/latex-insert-environment "itemize"))

(defun my/latex-figure ()
  "Insertar entorno figure."
  (interactive)
  (my/latex-insert-environment "figure"))

(defun my/latex-table ()
  "Insertar entorno table."
  (interactive)
  (my/latex-insert-environment "table"))

(with-eval-after-load 'latex
  (define-key LaTeX-mode-map (kbd "C-c e") #'my/latex-equation)
  (define-key LaTeX-mode-map (kbd "C-c f") #'my/latex-figure)
  (define-key LaTeX-mode-map (kbd "C-c t") #'my/latex-table)
  (define-key LaTeX-mode-map (kbd "C-c i") #'my/latex-itemize))

;; Exportar a DOCX con pandoc
(defun my/export-buffer-to-docx ()
  "Exportar el archivo actual a DOCX usando pandoc.
Funciona bien para Org y Markdown."
  (interactive)
  (unless buffer-file-name
    (user-error "El buffer no está asociado a un archivo"))
  (unless (executable-find "pandoc")
    (user-error "pandoc no está instalado"))
  (save-buffer)
  (let* ((in  (buffer-file-name))
         (out (concat (file-name-sans-extension in) ".docx")))
    (call-process "pandoc" nil "*Pandoc Output*" t in "-o" out)
    (message "Exportado a %s" out)))

(global-set-key (kbd "C-c e w") #'my/export-buffer-to-docx)
\end{verbatim}
\end{document}
